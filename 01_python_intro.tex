\documentclass[mathserif, xcolor=table, svgnames]{beamer}
\mode<presentation>
{
  \usetheme{Hannover}
  \setbeamercovered{transparent}
}
\usepackage[english]{babel}
\usepackage[utf8]{inputenc}
\usepackage[T1]{fontenc}
\usepackage{amsthm}
\usepackage{array,xspace}
\usepackage{dcolumn}
\usepackage{eulervm}
\usepackage{eurosym}
\usepackage{graphicx}
\input{isomath.tex}
\usepackage{booktabs, multicol, multirow}
\usepackage{listings}
\usepackage{relsize}
\usepackage{wasysym}

\colorlet{OurColor}{LawnGreen!40}
\colorlet{ShadedRowColor}{LightSkyBlue!40}

\newcommand{\individuals}{\mathbbm{I}}
\newtheorem{proposition}{Proposition}
\newcommand{\std}[1]{\small\color{lightgray}{#1}}
\long\def\GobbleColumnStart#1\GobbleColumnStop{}
\let\GobbleColumnStop\relax
\newcolumntype{i}{>{\GobbleColumnStart}c<{\GobbleColumnStop}}
\newcommand{\V}{\ensuremath{\surd}}
\graphicspath{{./}{cartoons/}{images/}}

\title[Python]{Introduction to Python}
\author{Ott Toomet}
\begin{document}
\lstset{language=Python}

\begin{frame}
  \maketitle
\end{frame}

\begin{frame}
  \tableofcontents
\end{frame}

\begin{frame}
\frametitle{Where We Stand}
\begin{columns}
  \begin{column}{0.5\linewidth}
    \begin{itemize}
    \item Introduction
    \item Methods
      \begin{itemize}
      \item \alert{python}, \alert{pandas}
      \item Linear algebra
      \item Probability and statistics
      \end{itemize}
    \item Guest speaker (Shawn Walker)
    \item Linear regression
      \begin{itemize}
      \item Causality
      \end{itemize}
    \item ML
      \begin{itemize}
      \item Experiment design
      \item Nearest neighbors
      \end{itemize}
    \end{itemize}
  \end{column}
  \begin{column}{0.5\linewidth}
    \begin{itemize}
    \item Methods
      \begin{itemize}
      \item Gradient Descent 
      \item Maximum Likelihood, logit
      \item regularization
      \end{itemize}
    \item ML
      \begin{itemize}
      \item Naive bayes
      \item PCA/dimensionality reduction
      \item Clusters \& recommenders
      \item Trees and forests
      \item Neural networks
      \end{itemize}
    \item Wrap-up
    \end{itemize}
  \end{column}
\end{columns}
\end{frame}

\section{Python}

\begin{frame}
  \frametitle{Python News}
  \begin{columns}[T]
    \begin{column}{0.6\linewidth}
        \includegraphics[width=\linewidth]{python_swallows_man.png}
    \end{column}
    \begin{column}{0.4\linewidth}
      Careful with Python!
    \end{column}
  \end{columns}
\end{frame}

\begin{frame}
  \frametitle{The Language}
  \url{https://www.slideshare.net/nowells/introduction-to-python-5182313}
\end{frame}

\section{Numpy}

\begin{frame}
  \frametitle{Numpy}
  \url{https://www.slideshare.net/PyData/introduction-to-numpy}
\end{frame}

\section{Pandas}

\begin{frame}[fragile]
  \frametitle{Series}
  Array with \emph{index}:
\begin{lstlisting}
In [7]: pd.Series([1, 2, 3, 4])
Out[7]: 
0    1
1    2
2    3
3    4
dtype: int64
\end{lstlisting}
  \begin{itemize}
  \item \emph{index}: $\sim$ row label
  \end{itemize}
\end{frame}

\begin{frame}[fragile]
  \frametitle{Series as dict}
\begin{lstlisting}
In [9]: pd.Series({ "WA" : 10, "CA" : 40, 
                    "NY" : 25, "AK" : 3})
Out[9]: 
AK     3
CA    40
NY    25
WA    10
dtype: int64
\end{lstlisting}
  \begin{itemize}
  \item index (row label) may be character
  \end{itemize}
\end{frame}


\begin{frame}[fragile]
  \frametitle{DataFrame}
  Dict of Series:
\begin{lstlisting}
In [10]: a = { "a" : [1,2],
   ....: "b" : ["x", "y"],
   ....: "c" : [True, False]}
In [12]: pd.DataFrame(a)
Out[12]: 
   a  b      c
0  1  x   True
1  2  y  False
\end{lstlisting}
  Note: common index (0,1) for all 3 Series
\end{frame}

\begin{frame}
  \frametitle{Pandas}
  \url{https://www.slideshare.net/maikroeder/pandas-16424935}
\end{frame}


\end{document}
