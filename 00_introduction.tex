\documentclass[mathserif, xcolor=table, svgnames]{beamer}
\mode<presentation>
{
  \usetheme{Hannover}
  \setbeamercovered{transparent}
}
\usepackage[english]{babel}
\usepackage[utf8]{inputenc}
\usepackage[T1]{fontenc}
\usepackage{amsthm}
\usepackage{array,xspace}
\usepackage{dcolumn}
\usepackage{eulervm}
\usepackage{eurosym}
\usepackage{graphicx}
\input{isomath.tex}
\usepackage{booktabs, multicol, multirow}
\usepackage{listings}
\usepackage{relsize}
\usepackage{wasysym}

\colorlet{OurColor}{LawnGreen!40}
\colorlet{ShadedRowColor}{LightSkyBlue!40}

\newcommand{\individuals}{\mathbbm{I}}
\newtheorem{proposition}{Proposition}
\newcommand{\std}[1]{\small\color{lightgray}{#1}}
\long\def\GobbleColumnStart#1\GobbleColumnStop{}
\let\GobbleColumnStop\relax
\newcolumntype{i}{>{\GobbleColumnStart}c<{\GobbleColumnStop}}
\newcommand{\V}{\ensuremath{\surd}}
\graphicspath{{./}{cartoons/}{images/}}

\title[INFX 574]{Data Science: Machine Learning \& Econometrics}
\author{Ott Toomet}
\begin{document}
\lstset{language=Python}

\begin{frame}
  \maketitle
\end{frame}

\begin{frame}
  \tableofcontents
\end{frame}

\section{Teachers}

\begin{frame}
  \frametitle{Me}
  \begin{columns}
    \begin{column}{0.5\linewidth}
      \begin{block}{}
        \includegraphics[width=0.9\linewidth]{ott.jpg}
      \end{block}
    \end{column}
    \begin{column}{0.5\linewidth}
      \begin{itemize}
      \item Economist/computational social scientist
        \begin{itemize}
        \item surveys, cellphone data, government registries
        \item statistical software development
        \item Social networks, segregation, causal analysis
        \end{itemize}
      \end{itemize}
    \end{column}
  \end{columns}
\end{frame}

\begin{frame}
  \frametitle{Srijan}
  \begin{columns}
    \begin{column}{0.5\linewidth}
      \begin{block}{}
        \includegraphics[width=0.9\linewidth]{srijan.jpg}
      \end{block}
    \end{column}
    \begin{column}{0.5\linewidth}
      \begin{itemize}
      \item 2nd year MA student at iSchool
      \end{itemize}
    \end{column}
  \end{columns}
\end{frame}

\begin{frame}
  \frametitle{ATT of Training}
  \includegraphics[width=0.9\linewidth]{11_sfs_att_employment_parttime.pdf}
\end{frame}

\begin{frame}
  \frametitle{Call Activity in Afghanistan}
  \includegraphics[width=1.2\linewidth]{pca_2015-02-01.png}
\end{frame}

\section{Objectives and Rules}


\begin{frame}
  \frametitle{Course Objectives}
  \includegraphics[width=\linewidth]{prove_or_disprove.png}
\end{frame}

\begin{frame}
  \begin{itemize}
  \item Get an overview of common econometric and machine learning
    methods 
  \item Become familiar with python (numpy)
  \item Understand, design, and critique statistical methods for
    common data types
  \item Know the usage, advantages and disadvantages of supervised and
    unsupervised methods
  \item Implement selected algorithms
  \end{itemize}
\end{frame}

\begin{frame}
  \frametitle{And Finally---Know the Limits of Data Science}
  \includegraphics[width=\linewidth]{what-chance-does-data-have.png}
\end{frame}

\begin{frame}
  \frametitle{Requirements}
  \begin{itemize}
  \item Be reasonably familiar with programming
    \begin{itemize}
    \item Python knowledge is not necessary
    \item \alert{but pick it up quickly!}
    \end{itemize}
  \item Basic knowledge of probability and statistics, matrix algebra
    \begin{itemize}
    \item A large chunk of machine learning are statistical models
    \item Matrices is the way to handle data in code (and in theory ;-)
    \end{itemize}
  \item INFX 573 should give you most of this background
  \end{itemize}
\end{frame}

\begin{frame}
  \frametitle{Grading}
  \begin{itemize}
  \item 5 Problem sets (70\%), 
  \item 2 Quizzes/mini-assignments (15\%)
  \item 9 Labs, participation (15\%)
  \item Assignemnts:
    \begin{itemize}
    \item Submit by deadline!
    \item Normally on canvas
    \item Late = points deducted (0.5 pt per hour late).
    \item Do it yourself
    \end{itemize}
  \end{itemize}
\end{frame}


\begin{frame}
  \frametitle{Calendar}
  \begin{itemize}
  \item Classes Tue, Thu 3:30--5:20 MGH 251
  \item Assignments due (by 3:30pm):
    \begin{itemize}
    \item April 13 (python, pandas)
    \item April 27 (linear regression, causality)
    \item May 9 (kNN, cross-validation)
    \item May 23 (gradient descent, maximum likelihood)
    \item June 2 (naive Bayes, PCA)
    \end{itemize}
  \item Quizzes
    \begin{itemize}
    \item April 18
    \item May 11
    \item June 1
    \end{itemize}
  \item Attend research seminar
  \end{itemize}
\end{frame}

\begin{frame}
  \frametitle{Schedule}
\end{frame}

\begin{frame}
  \frametitle{Textbooks}
  
\end{frame}


\section[Topic]{What Is Data Science?}

\begin{frame}
  \frametitle{What Is Data Science?}
  \begin{center}
    \includegraphics[width=\linewidth]{fortune_teller_predictive_analyst.jpg}
  \end{center}
\end{frame}


\begin{frame}
\frametitle{Data}
\begin{itemize}
\item Information stored on computers and what we can analyze
\item Collecting data \emph{much} easier than earlier
\item $\Rightarrow$ Big Data: cannot be easily analyzed with the off-the-shelf
  software
  \begin{itemize}
  \item large volume (>TB)
    \begin{itemize}
    \item Satellite images, Walmart prices
    \end{itemize}
  \item Speed of production
    \begin{itemize}
    \item CDR, stock prices, social media, CERN
    \end{itemize}
  \item Variety: text, video, images, ...
    \begin{itemize}
    \item Amazon, Google
    \item Usually varying quality
    \end{itemize}
  \end{itemize}
\end{itemize}
\end{frame}

\begin{frame}
\frametitle{Data Science}
How to handle such data
\begin{enumerate}
\item Define the (exact) problem
  \begin{itemize}
  \item (Social) science
  \end{itemize}
\item Find and explore the relevant data
  \begin{itemize}
  \item Computer, visualization literacy
  \end{itemize}
\item Design analytic framework
  \begin{itemize}
  \item Statistics
  \end{itemize}
\item Implement analysis
  \begin{itemize}
  \item Programming, algorithms, software
  \end{itemize}
\item Communicate
  \begin{itemize}
  \item Visualization, communication
  \end{itemize}
\end{enumerate}
\end{frame}


\begin{frame}
  \frametitle{Skills}
  A good data scientist is
  \begin{itemize}
  \item (Social) scientist: can identify meaningful and feasible
    problems 
  \item Computer scientist: knows algorithms, data structures,
    security, technological solutions
  \item Programmer: can implement the algorithms
  \item Hacker: can squeeze a lot out of computers
  \item PR consultant: can tell a story
  \item Lawyer: knows regulations around privacy and security
  \end{itemize}
\end{frame}


\begin{frame}
  \frametitle{Computer Skills}
  \begin{itemize}
  \item Programming
    \begin{itemize}
    \item Regular expressions
    \item Scripting: bash, python, R, php, perl, \dots
    \item General: C(++), java, scala, \dots
    \item Data structures: maps, sets, graphs, \dots
    \end{itemize}
  \item Statistics
    \begin{itemize}
    \item \alert{Linear algebra, calculus, inference}, \dots
    \item \alert{Machine learning, predictive modeling}, \dots
    \item Network analysis
    \item Statistical software: R, \alert{numpy, scipy}, stata, matlab, \dots
    \end{itemize}
  \item Scalable data management
    \begin{itemize}
    \item hadoop, mapreduce, spark, \dots
    \item sql, nosql, Hive, \dots
    \end{itemize}
  \item Visualization
    \begin{itemize}
    \item ggplot, D3, tableau, \dots
    \end{itemize}
  \end{itemize}
\end{frame}


\section{Test}

\begin{frame}
  \begin{itemize}
  \item True or False (or ``don't know'')?
  \end{itemize}
  \begin{equation*}
    \begin{bmatrix}
      1 & 2\\
    \end{bmatrix}
    \cdot
    \begin{bmatrix}
      3 \\ 4\\
    \end{bmatrix}
    = 
    \begin{bmatrix}
      3 & 4 \\
      6 & 8 \\
    \end{bmatrix}?
  \end{equation*}
\end{frame}

\begin{frame}
  \frametitle{Determinant}
  \begin{equation*}
    \left|
    \begin{bmatrix}
      1 & 2 \\
      3 & 4 \\
    \end{bmatrix}
    \right|
    = -2?
  \end{equation*}
\end{frame}

\begin{frame}
  \frametitle{Eigenvalues}
  Eigenvalues of
  \begin{equation*}
    \begin{bmatrix}
      1 & 0 \\
      0 & 2 \\
    \end{bmatrix}
  \end{equation*}
  are
  \begin{equation*}
    1 \qquad\text{and}\qquad 1/2?
  \end{equation*}
\end{frame}

\begin{frame}
  \frametitle{Expected Value}
  Expected value of a standard normal random variable is 1
\end{frame}

\begin{frame}
  \frametitle{Distribution Function}
  Which graph is \emph{cumulative distribution function} of a standard
  uniform random variable?
  \includegraphics[width=0.9\linewidth]{uniform.pdf}
\end{frame}

\begin{frame}
  \frametitle{Derivative}
  Let
  \begin{equation*}
    f(u, \kappa) = a + \pi\, u + 2\kappa u^{4} + \log u
  \end{equation*}
  Now
  \begin{equation*}
    \pderiv[f]{\kappa} = \pi + 8\kappa u^{3} - \frac{1}{u}
  \end{equation*}
\end{frame}


\section[ML]{Machine Learning}

\begin{frame}
  \frametitle{Learn!}
  \begin{columns}
    \begin{column}{0.5\linewidth}
      \includegraphics[width=0.9\linewidth]{class_a.jpg}
    \end{column}
    \begin{column}{0.5\linewidth}
      \includegraphics[width=0.9\linewidth]{class_b.jpg}
    \end{column}
  \end{columns}
\end{frame}

\begin{frame}
  \frametitle{Predict!}
  \includegraphics[width=0.7\linewidth]{class_todo.jpg}
\end{frame}

\begin{frame}
  \frametitle{Generalization Counts}
  \begin{itemize}
  \item We observe a very thiny sample of the potential patterns
    \begin{itemize}
    \item Example: 10k words in dictionary $\Rightarrow$
      $2^{10,000} \approx 10^{3,000}$ different combinations
    \item Compare with:
      \begin{itemize}
      \item particles in the visible universe: $\sim 10^{90}$ 
      \item  age of Universe: $\sim \cdot 10^{18}$s
      \end{itemize}
    \item Still have to classify spam/non-spam
    \end{itemize}
  \end{itemize}
\end{frame}



\end{document}
